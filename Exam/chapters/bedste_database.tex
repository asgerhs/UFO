Der er mange situationer hvor en SQL database ville være mere relevant end noSQL, 
det kommer virkelig an på hvordan dit system fungerer og hvad det skal udføre. Spørgsmålet om hvornår NoSQL er bedst er ikke lige så let at svare altid. 
Men der er nogle klare fordele ved NoSQL, som bestemt appellerer til nogle behov der ikke kan, 
eller i mindste fald er svært, at tilfredsstille i SQL. 
At tage en beslutning om man skal bruge den ene type frem for den anden, 
bør være en beslutningstagning baseret på behov. Mere præcist hvilke data der skal lagres, 
hvor stor en mængde af data der skal lagres samt hvilke behov man har vedr. skalerbarhed, nedetid og omkostninger.\newline

NoSQL er, som beskrevet tidligere, stærk når det kommer til at skalere grundet dets fleksibilitet. 
Dette er en større udfordring hos SQL databaser på grund af den meget rigide tilgang der ses i relationsdatabaser. 
Hvis der sker en større dataændring, f.eks. kan det være en udvidelse af produkter på en hjemmeside. 
I SQL ville man stå overfor et stort stykke arbejde da relationer og tabeller skal opdateres, 
tilføjes og fjernes. Et problem som NoSQL ikke støder ind i. Dernæst skal der tages et kig på tilgængeligheden af data for de to databasetyper. 
Som nævnt tidligere er en af NoSQL’s styrker hvor let det er at skalere selve databasen. 
På grund af skalering foregår horisontalt ved at tilføje flere maskiner og fordele byrden ud, 
kan man undgå downtime. SQL derimod, har svært ved at skalere horisontalt, 
hvorfor det må forventes at der vil forekomme downtime hvis en SQL database skal opgraderes eller ændres.\newline

Et område hvor NoSQL er særligt appellerende er når man skal administrere big data. 
Banalt sagt er big data enorme mængder af data, både ustruktureret, semi-struktureret og struktureret. 
På grund af at dataen kommer i så mange former, er det afgørende at have en fleksibel database, 
som kan administrere de forskellige typer uden problemer. Ydermere gør skaleringen read performance bedre, 
ved at have flere servere, der kan håndtere trafikken.
Store virksomheder bruger flere forskellige typer af databaser til forskellige programmer. \newline
